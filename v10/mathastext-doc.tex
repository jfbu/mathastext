\documentclass[a4paper]{article}

\makeatletter
\def\sbverb{\makeatletter\def\verbatim@font{\usefont{T1}{lmtt}{sb}{n}}\makeatother}
\def\bverb{\makeatletter\def\verbatim@font{\usefont{T1}{lmtt}{b}{n}}\makeatother}
\def\LMvtt@scale{1.12}
\def\LMtt@scale{1.12}
\makeatother

\sbverb

\usepackage{shortvrb}
\MakeShortVerb{\|}

\usepackage{amsmath}
\usepackage[hscale=0.7,vscale=0.7]{geometry}

\usepackage[latin1]{inputenc}
\usepackage[T1]{fontenc}

\usepackage[nomath,lighttt,variablett]{lmodern}

\renewcommand{\familydefault}{\ttdefault}

\usepackage[alldelims,symbolmax,eulergreek,mathaccents]{mathastext}

\MathastextDeclareVersion{lmvtt}{T1}{lmvtt}{m}{n}
\MathastextDeclareVersion{lmvttmedium}{T1}{lmvtt}{sb}{n}
\MathastextDeclareVersion{times}{T1}{ptm}{m}{n}
\MathastextDeclareVersion{palatino}{T1}{ppl}{m}{n}
\MathastextDeclareVersion{bookman}{T1}{pbk}{m}{n}
\MathastextDeclareVersion{charter}{T1}{bch}{m}{n}
\MathastextDeclareVersion{newcent}{T1}{pnc}{m}{n}
\MathastextDeclareVersion{utopia}{T1}{put}{m}{n}
\MathastextDeclareVersion{helvetica}{T1}{phv}{m}{n}
\MathastextDeclareVersion{avant}{T1}{pag}{m}{n}
\MathastextDeclareVersion{courier}{T1}{pcr}{m}{n}
%% \MathastextDeclareVersion{symb}{U}{psy}{m}{n}
%% \MathastextDeclareVersion{euler}{U}{zeur}{m}{n}

\linespread{1.12}
\begin{document}

\title{The |mathastext| package}
\author{Jean-Fran�ois B. \\ 2589111+jfbu@users.noreply.github.com}
\date{January 25, 2011}

\maketitle

\begin{abstract}
   The |mathastext| package\footnote{This documentation describes
   |mathastext| version 1.0 and was last revised on 2011/01/25.}
   propagates the document {\em text} font to
   {\em mathematical} mode, for the letters of the Latin
   alphabet and, optionally, some further ASCII-127
   characters. The idea is to produce handouts or research
   papers with a less book-like typography than what is
   typical of standard \TeX\ with the Computer Modern
   fonts. Hopefully, this will force the reader to
   concentrate more on the contents ;-). It also makes it
   possible (for a document with simple mathematics) to use
   a quite arbitrary font without worrying too much that it
   does not have specially designed accompanying math
   fonts. Also, |mathastext| provides a simple mechanism in
   order to use many different choices of (text hence, now,
   math) fonts in the same document (not that we recommend
   it!).  A final aspect is that |mathastext| helps produce
   smaller PDF files.
\end{abstract}

\section{Introduction}

This document has
|\usepackage{mathastext}| (with some options described later) in its
preamble. Let us now typeset some mathematics.

\long\def\mathA%
{\begingroup\narrower\narrower
    We define the quantities $X = X(\nu,n,a)$ and $Y(\nu,n,a)$ by the following
    expressions:
  \begin{align}
      X &:= \frac{g_{\nu+1}}{h_{\nu+1}} = \frac1a \frac{e_\nu}{f_\nu}\;,\\
      Y &:= - \frac1a \frac{c^{\nu+1}_n}{d^{\nu+1}_n} = - \frac{c_1^\nu}{d^\nu_1}\;.
      \end{align}
From Proposition 2.7.1 we have 
$\mu_\nu = \frac{2a^{\frac12}}{1 - a^2}\left(-ah_{\nu+1} c_1^\nu  + g_{\nu+1} d_1^\nu
  \right)$ and using
the first two relations in Proposition 5.9.5 gives:
\begin{equation}
  \mu_\nu = \frac{2n a}{1 - a^2} \frac{-h_{\nu+1} c_n^{\nu+1} + g_{\nu+1}
    d_n^{\nu+1}}{h_{\nu+1} d_n^{\nu+1} - g_{\nu+1} c_n^{\nu+1}} = \frac{2n
    a}{1 - a^2} \frac{X+aY}{1+aXY}\;.
\end{equation}\endgroup}

%\let\mathA\relax
\mathA

The text font of the present document is the variable-width typewriter face
from Latin Modern (in its ``light'' variant): the same font is used in
mathematical mode for the Latin letters, so that the distinction between the
text and the equations is blurred out. This will shock some, but as a
mathematician I consider the text itself to be also an equation. So, I never
quite adhered to all this business of using oblique or italic letters in
inline or displayed mathematics.\footnote{originally, {\bfseries everything}
  was oblique, as mathematics was handwritten. Why should we use upright
  letters to say ``group'' and oblique letters for an element \textit{g} of
  the group \textit{G}? admittedly in French mathematical typography capital
  Roman letters are upright, but I want also the lowercase to follow suit.}

The option |eulergreek| was passed to |mathastext| to use the Euler font from
the A.M.S. There is a similar option |symbolgreek| to use the Postscript
Symbol font (included in the core \LaTeX2e\ distribution) to typeset the Greek
letters.  Both are upright. If no option is passed the Greek letters will be
in the mathematical ``letters'' font (by default Computer Modern), and they
will be slanted,\footnote{but see the mention of the |fourier| package
  \textit{infra}.} which typically is not suitable for documents using
|mathastext| (they will also generally look very thin comparatively to the
text font).


Another thing to point out is that the subscript sizes have been dramatically
increased, so that the text looks almost as if it had been written on a
typewriter (an extremist user would use the \LaTeX\ command
|\DeclareMathSizes| to let subscripts have the same size as the main text
font, we chose a middle-ground between this extreme and the default of
\LaTeX). The option |defaultmathsizes| tells the package not to apply any
changes to  the
\LaTeX\ defaults.

\MathastextVersion{lmvttmedium}

The package provides a simple mechanism, using the \LaTeX\ ``math versions'',
to change the design of the  mathematics in the same document.\footnote{the
  present document also illustrates how quickly one gets a headache if one
  tries to read a document with multiple font changes\dots} For example, we
made in the preamble the declaration:\\
\hbox to \hsize{\hss |\MathastextDeclareVersion{lmvttmedium}{T1}{lmvtt}{sb}{n}|\hss}\\
If we now issue the command |\MathastextVersion{lmvttmedium}| the looks of
mathematics change:

\long\def\mathB{\begingroup\narrower\narrower Similarly from 8.2.8 and 1.8.2, 
      $\mu_{\nu+1} = \frac{2a^{\frac12}}{1-a^2}\left(- h_{\nu+1} c_1^\nu + a
        g_{\nu+1} d_1^\nu \right)$ which gives using Proposition 3.1.4 and
      Definition 1.5.9:
      \begin{equation}
        \mu_{\nu+1} = \frac{2n
          a}{1 - a^2} \frac{aX+Y}{1+aXY}\;.
      \end{equation}\endgroup}

%\let\mathB\relax
\mathB

\noindent Let us try the more radical |\MathastextVersion[utopia]{utopia}|. The ``utopia'' math version was defined in the preamble via\\
\hbox to \hsize{\hss |\MathastextDeclareVersion{utopia}{T1}{put}{m}{n}|\hss}

\MathastextVersion[utopia]{utopia}

\mathB

\noindent 
The optional argument to |\MathastextVersion| tells it to change also the text
font. While we are at considering the Utopia font, let us mention that it would have been better however to use the |fourier|
package\footnote{of \textsc{Michel~Bovani}.}, which has an upright Greek alphabet which
would look better here in combination with the utopia font than the Euler
font. So, there is also the possibility to put in the preamble:
\begin{verbatim}
\usepackage[upright]{fourier}
\usepackage{mathastext}
\end{verbatim}
in order to gain access to the Greek letters via the |fourier| package. 
  If we now want to keep |utopia| as text font and
foolishly typeset our mathematics
in \hbox{\MathastextVersion[avant]{avant}Avant Garde}, we
certainly can using
 |\MathastextVersion{avant}|:\\
\MathastextVersion{avant}

\mathB

\noindent Why stop here? we issue a 
|\MathastextVersion[courier]{avant}|:

\MathastextVersion[courier]{avant}

\mathB

\hyphenchar\font=45 We note that the Greek letters from the Euler font look a
bit small compared to \hbox{\MathastextVersion[avant]{avant}Avant Garde
  letters}. The
command\\
\hbox to \hsize{\hss |\MathastextEulerScale{1.1}|\hss} in the preamble would
have scaled the Euler font to be 10\%\ larger than its nominal size. This is a
preamble-only command and we didn't use it here as we started with the Latin
Modern font |lmvtt|.\footnote{%
It is in fact |lmvtt| itself that we have scaled to
112\%\ of nominal size. |mathastext| provides no command for this,
  one has to include in the preamble a suitably modified |t1lmvtt.fd| file, on
  the model of what one sees for example in |t1phv.fd| from the \LaTeX2e\ core
  package |psnfss|.} 
The |mathastext| package loads the Euler font via the
|uzeur.fd| font definition file from the |eulervm| package\footnote{of
  \textsc{Walter~Schmidt}.} and this makes a scaling mechanism available (the
package is not loaded but the file |uzeur.fd| must be accessible to
\LaTeX). A similar command |\MathastextSymbolScale| is also provided (no
specific package needed). The Euler font has the advantage compared to
Symbol to have a bold version: {\boldmath$\alpha\beta\gamma$}.

\bverb

\hbox to \hsize{\hss |\MathastextVersion[charter]{lmvttmedium}|\hss}\par

\MathastextVersion[charter]{lmvttmedium}

\medskip

\mathB



The package |mathastext| can be loaded with the option |basic|. The |basic|
option tells the package to limit its scope to the Latin letters and to the
digits. Without this option the default is to treat also
the characters:\DeleteShortVerb{\|}\\
\centerline{text: !\,?\,*\,,\,.\,:\,;\,+\,-\,=\,(\,)\,[\,]\,/\,\#\,\$\,\%\,\&\,<\,>\,|\,\{\,\}\,\char92}
Due to their non-availability in the OT1-encoding (apart from typewriter
fonts), the characters {<\,>\,|\,\{\,\} and \char92} \MakeShortVerb{\|} need to be
specifically requested via the option |alldelims|. In the present document we
used the option |symbolmax| so that in fact the characters listed above (apart
from
\#\,\$\,\%\,\& and \char92\,) are typeset in mathematical mode using the Symbol font:\\
\DeleteShortVerb{\|}
\centerline{math: $!\,?\,*\,,\,.\,:\,;\,+\,-\,=\,(\,)\,[\,]\,/\,\#\,\$\,\%\,\&\,<\,>\,|\,\{\,\}\,\backslash$}
\MakeShortVerb{\|}
Besides the option |basic| which cancels
consideration of all these characters, various subsets can be excluded, via
the options |noequal|, |noplusnominus|,
|noexclam|, |nospecials|, |nodigits|,  |noparenthesis|, |nopunctuation|, |noasterisk|.


Regarding the Greek letters, we already mentioned |symbolgreek| and
|eulergreek|, and there is also |selfGreek| which is provided for OT1-fonts
which have the capital Greek letters.

Other self-explanatory options are |symboldigits| and |eulerdigits|.

The option |mathaccents| tells |mathastext| to also pick up the mathematical
accents from the text font; the package makes a choice between OT1 and T1 slot
positions: if not all fonts declared to |mathastext| are OT1-encoded, then the
T1-positions are assumed. A similar approach is taken for |\imath| and
|\jmath| (commands |\inodot| and |\jnodot| are defined to use the text font,
and the identification with |\imath| and |\jmath| can be prevented via the
option |\defaultimath|).


There is also the issue of the |\vec| accent. As it looks awful on upright
letters (its arrowhead is slanted, and it is shifted to the right), the
default is to take the glyph from the mathematical font provided by the
|fourier| package: $\vec\imath, \vec\jmath, \vec k$ (it is not needed to load
the |fourier| package for this). The option |defaultvec| tells |mathastext|
not to overwrite the standard |\vec|. Furthermore the package always makes available a ``poor man'''s
|\pmvec|: $\pmvec\imath, \pmvec\jmath, \pmvec k$.


\newcommand\ttprint[1]{|\char92#1|~$\csname #1\endcsname$}

The option |symbolre| 
\sbverb
gives access to the three symbols \ttprint{Re}, \ttprint{Im},
\ttprint{DotTriangle}, 
\bverb
and the option |symbolmisc| to the following list:\par
\sbverb
{\narrower\narrower
\noindent{\ttprint{prod}}\ 
{\ttprint{sum}}\ 
{\ttprint{implies}}\ 
{\ttprint{impliedby}}\ 
{\ttprint{iff}}\ 
{\ttprint{shortiff}}\ 
{\ttprint{to}}\ 
{\ttprint{longto}}
{\ttprint{mapsto}}\ 
{\ttprint{longmapsto}}\ 
%
{\ttprint{aleph}}\ 
{\ttprint{inftypsy}}\ 
{\ttprint{emptyset}}\ 
{\ttprint{surd}}\ 
{\ttprint{nabla}}\ 
{\ttprint{angle}}\ 
{\ttprint{forall}}\ 
{\ttprint{exists}}\ 
{\ttprint{neg}}\ 
{\ttprint{clubsuit}}\ 
{\ttprint{diamondsuit}}\ 
{\ttprint{heartsuit}}\ 
{\ttprint{spadesuit}}\ 
{\ttprint{smallint}}\ 
{\ttprint{wedge}}\ 
{\ttprint{vee}}\ 
{\ttprint{cap}}\ 
{\ttprint{cup}}\ 
{\ttprint{bullet}}\ 
{\ttprint{div}}\ 
{\ttprint{otimes}}\ 
{\ttprint{oplus}}\ 
{\ttprint{pm}}\ 
{\ttprint{ast}}\ 
{\ttprint{times}}\ 
{\ttprint{proptopsy}}\ 
{\ttprint{mid}}\ 
{\ttprint{leq}}\ 
{\ttprint{geq}}\ 
{\ttprint{approx}}\ 
{\ttprint{supset}}\ 
{\ttprint{subset}}\ 
{\ttprint{supseteq}}\ 
{\ttprint{subseteq}}\ 
{\ttprint{in}}\ 
{\ttprint{sim}}\ 
{\ttprint{cong}}\ 
{\ttprint{perp}}\ 
{\ttprint{equiv}}\ 
{\ttprint{notin}}\ 
{\ttprint{langle}}\ 
{\ttprint{rangle}}
\par}
\bverb

\noindent The idea of the |symbolmisc| option (together with |alldelims|) is
to (perhaps) reduce for documents with simple mathematics the number of fonts
used. The $\inftypsy$ symbol is small (presumably as it is designed already at
script-size) compared to the default $\infty$, so we gave it another name and
there is the possibility to do |\renewcommand{\infty}{\inftypsy}|.\footnote{The user can also
  give a try to
  {\usefont{T1}{lmtt}{b}{n}\char92renewcommand\{\char92int\}\{\char92smallint\}}.}

\hbox to \hsize{\hss|\MathastextVersion[lmvtt]{normal}|\hss}

\MathastextVersion[lmvtt]{normal}
\sbverb
\section{Usage}

\subsection{Basic}

One adds |\usepackage{mathastext}| after the other font-related |\usepackage|: this will set up the Latin letters and digits (and further
characters, as described \textit{supra}) to be typeset with the document text
font also in mathematical mode.  The command |\Mathastext| can be issued
anywhere in the preamble to reinitialize |mathsastext| if the
|\encodingdefault|, |\familydefault|, |\seriesdefault|, or
|\shapedefault| has been changed. Alternatively one can use:\\
\hbox to \hsize{\hss|\MathastextWillUse{<enc>}{<fam>}{<ser>}{<sh>}|\hss}\\
It is also possible to use one or more of the commands
|\Mathastextencoding{<enc>}|, |\Mathastextfamily{<fam>}|,
|\Mathastextseries{<ser>}|, |\Mathastextshape{<sh>}|, followed by a mandatory
|\Mathastext| command (which has no argument).

The command |\Mathastextboldvariant{<var>}| specifies the series (b,bx,sb,
etc\dots) to be used by
|\mathbf|, and after |\boldmath|.

\subsection{Using math versions}

\LaTeX\ has two mathematical versions: |normal| and |bold|. The command\\
\hbox to \hsize{\hss|\MathastextDeclareVersion{}{}{}{}|\hss} 
allows to declare arbitrarily many (well
in practice not more than a dozen or so distinct) |versions|. As an example, let us reproduce here
the declarations from this document:
\begin{verbatim}
\MathastextDeclareVersion{lmvtt}{T1}{lmvtt}{m}{n}
\MathastextDeclareVersion{lmvttmedium}{T1}{lmvtt}{sb}{n}
\MathastextDeclareVersion{times}{T1}{ptm}{m}{n}
\MathastextDeclareVersion{palatino}{T1}{ppl}{m}{n}
\MathastextDeclareVersion{bookman}{T1}{pbk}{m}{n}
\MathastextDeclareVersion{charter}{T1}{bch}{m}{n}
\MathastextDeclareVersion{newcent}{T1}{pnc}{m}{n}
\MathastextDeclareVersion{utopia}{T1}{futs}{m}{n}
\MathastextDeclareVersion{helvetica}{T1}{phv}{m}{n}
\MathastextDeclareVersion{avant}{T1}{pag}{m}{n}
\MathastextDeclareVersion{courier}{T1}{pcr}{m}{n}
\end{verbatim}
These commands have to be issued in the preamble of the document. Then in the
body of the document one uses the standard \LaTeX\ command
|\mathversion{}|. For example after |\mathversion{bookman}|, the Bookman font
will be used for letters and digits in mathematics.\footnote{This mechanism is
  mainly provided to allow some context dependent design changes, for example
  for the statements of theorems or definitions.}

Alternatively one can use |\MathastextVersion{bookman}| which is exactly the
same thing, except that it allows an optional argument which  enforces also a
change of the text font. This has been described above.

\subsection{Package options}

\begingroup
\parindent0pt

|basic|: only letters and digits. The default is to also change the font for
!\,?\,*\,,\,.\,:\,;\,+\,-\,=\,(\,)\,[\,]\,/\,\#\,\$\,\%\,\&. Various options
exclude some subsets: |noexclam|\  !\,?\  |noasterisk|\ *\  |nopunct|\ ,\,.\,:\,;\  
|noplusnominus|\ +\,- |noequal|\ =\  |noparenthesis|\, (\,)\,[\,]\,/
\  |nospecials|\ \#\,\$\,\%\,\& and |nodigits|. The |alldelims| option (not suitable for
OT1-encoding) adds the characters \DeleteShortVerb{\|} <\,>\,|\,\{\,\}\,$\backslash$
\MakeShortVerb{\|} to the default list.\par

\medskip

|symbolgreek|, |symboldigits|: to let Greek letters (digits) use the Symbol
font. |eulergreek|, |eulerdigits|: to let Greek letters (digits) use the Euler
font.

\medskip


|mathaccents|: accents $\acute{a}\,
\grave a\,
\ddot a\,
\tilde a\,
\bar a\,
\breve a\,
\check a\,
\hat a\,
\dot a\,
\mathring a$ will use the text font.

\medskip

|symbolre| and |symbolmisc| instruct to use the Symbol font for some
miscellaneous symbols (described \textit{supra}). The option |symbol|
combines |symbolre|, |symbolgreek| and |symbolmisc|. The option |symbolmax|
extends |symbol| as it will also typeset all characters described above (others than letters and digits)
in the Symbol font (except the specials \#\,\$\,\%\,\& and the backslash $\backslash$).

\medskip
\endgroup
The package defines commands |\Mathnormal|, |\Mathrm|, and |\Mathbf|, and it
will 
overwrite |\mathnormal|, |\mathrm|, |\mathbf| except if the options
|defaultnormal|, |defaultrm|, or |defaultbf| are passed.
The package defines |\inodot| and |\jnodot|\footnote{but the $\jnodot$ glyph
   is often unavailable in non-\TeX\ fonts.} and it will overwrite |\imath|
and |\jmath| except if option |defaultimath| is passed.

The remaining options |selfGreek| (only
suitable for OT1-encoded fonts), |defaultvec|, and |defaultmathsizes| have
been described before.

\medskip All the options passed to the package apply globally to all math
versions (if any) declared by |\MathastextDeclareVersion|.  The commands (only
for the preamble) |\MathastextSymbolScale{<x>}| and
|\MathastextEulerScale{<x>}| will scale the respective fonts by factor <x>.

We used\\
\hbox to \hsize{\hss
  |\usepackage[alldelims,symbolmax,eulergreek,mathaccents]{mathastext}|\hss}\\
in the preamble of the \TeX\ source of this file.
\end{document}

